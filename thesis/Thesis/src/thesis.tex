\documentclass[a4paper,english]{ifimaster}

\usepackage[utf8]{inputenc}
\usepackage{babel,duomasterforside}
\usepackage{hyperref}
\usepackage{pdfsync}
\usepackage{csquotes}
\usepackage{minted}
\usepackage{varioref}
\usepackage[backend=biber]{biblatex}

\addbibresource{citations.bib}

\newcommand{\todo}[1]{\textcolor{red}{[[TODO: #1]]}\PackageWarning{TODO:}{#1!}}

\title{Three Way Merge for Feature Model Evolution Plans}
\date{May 2021}
\author{Eirik Halvard Sæther}
\synctex=1

\begin{document}
\duoforside[dept={Department of Informatics},
program={Informatics: Programming and Systems Architecture},
long]

\frontmatter{}

\chapter*{Acknowledgements}

Thanks to Ida

Thanks to Ingrid and Crystal

Thanks to Germans, everyone on the LTEP project for input etc.

\chapter*{Abstract}

\todo{write abstract}

Feature Model Evolution Plans is intended to help ease the development of software product lines (SPLs). Feature Models allow software engineers to explicitly encode the similarities and differences of an SPL. However, due to the changing nature of an SPL, Evolution Plans allows for representing the \textit{evolution} of a feature model, not just the feature model as a single point in time.

Evolution planning of an SPL is often a dynamic, changing process, due to changing demands of the focus of development. The evolution planning is often not just done by a single engineer, but multiple engineers, working separately and independent of each other. Due to these factors, the need to unify and synchronize the changes the evolution plan emerges.

In this thesis, we develop a merge tool for Feature Model Evolution Plans. The core of the tool is a three-way merge algorithm. Given two different versions of an evolution plan, together with the common evolution plan they were derived from, the merge algorithm will attempt to merge all the different changes from both versions. If the merges are unifiable, the algorithm will succeed and yield the merged result containing the changes from both versions. However, if the changes are conflicting in any way, breaking the structure or semantics of evolution plans, the algorithm will stop, telling the user the reason of failure.

\tableofcontents{}
\listoffigures{}
\listoftables{}

\chapter*{Preface}

\todo{write better and more}
something about the LTEP project

something about summer project?

\mainmatter{}
\chapter{Introduction}%
\label{cha:introduction}

\section{Chapter Overview}%
\label{sec:chapter_overview}

\section{Project Source Code}%
\label{sec:project_source_code}

\chapter{Background}%
\label{cha:background}

\chapter{Formal Semantics of Feature Model Evolution Plans}%
\label{cha:formal_semantics_of_feature_model_evolution_plans}

\chapter{Three Way Merge Algorithm}%
\label{cha:three_way_merge_algorithm}

\section{Algorithm Overview}%
\label{sec:algorithm_overview}

\section{Converting To a Suitable Representation}%
\label{sec:converting_to_a_suitable_representation}

\section{Detecting the Changes Between Versions}%
\label{sec:detecting_the_changes_between_versions}

\section{Merging Intended Changes}%
\label{sec:merging_intended_changes}

\section{Ensuring structural and semantic soundness of merge result}%
\label{sec:ensuring_structural_and_semantic_soundness_of_merge_result}

\chapter{Conclusion and Future Work}%
\label{cha:conclusion_and_future_work}

\backmatter{}

\printbibliography

\end{document}
